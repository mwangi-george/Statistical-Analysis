% Options for packages loaded elsewhere
\PassOptionsToPackage{unicode}{hyperref}
\PassOptionsToPackage{hyphens}{url}
\PassOptionsToPackage{dvipsnames,svgnames,x11names}{xcolor}
%
\documentclass[
  letterpaper,
  DIV=11,
  numbers=noendperiod]{scrartcl}

\usepackage{amsmath,amssymb}
\usepackage{lmodern}
\usepackage{iftex}
\ifPDFTeX
  \usepackage[T1]{fontenc}
  \usepackage[utf8]{inputenc}
  \usepackage{textcomp} % provide euro and other symbols
\else % if luatex or xetex
  \usepackage{unicode-math}
  \defaultfontfeatures{Scale=MatchLowercase}
  \defaultfontfeatures[\rmfamily]{Ligatures=TeX,Scale=1}
\fi
% Use upquote if available, for straight quotes in verbatim environments
\IfFileExists{upquote.sty}{\usepackage{upquote}}{}
\IfFileExists{microtype.sty}{% use microtype if available
  \usepackage[]{microtype}
  \UseMicrotypeSet[protrusion]{basicmath} % disable protrusion for tt fonts
}{}
\makeatletter
\@ifundefined{KOMAClassName}{% if non-KOMA class
  \IfFileExists{parskip.sty}{%
    \usepackage{parskip}
  }{% else
    \setlength{\parindent}{0pt}
    \setlength{\parskip}{6pt plus 2pt minus 1pt}}
}{% if KOMA class
  \KOMAoptions{parskip=half}}
\makeatother
\usepackage{xcolor}
\setlength{\emergencystretch}{3em} % prevent overfull lines
\setcounter{secnumdepth}{-\maxdimen} % remove section numbering
% Make \paragraph and \subparagraph free-standing
\ifx\paragraph\undefined\else
  \let\oldparagraph\paragraph
  \renewcommand{\paragraph}[1]{\oldparagraph{#1}\mbox{}}
\fi
\ifx\subparagraph\undefined\else
  \let\oldsubparagraph\subparagraph
  \renewcommand{\subparagraph}[1]{\oldsubparagraph{#1}\mbox{}}
\fi

\usepackage{color}
\usepackage{fancyvrb}
\newcommand{\VerbBar}{|}
\newcommand{\VERB}{\Verb[commandchars=\\\{\}]}
\DefineVerbatimEnvironment{Highlighting}{Verbatim}{commandchars=\\\{\}}
% Add ',fontsize=\small' for more characters per line
\usepackage{framed}
\definecolor{shadecolor}{RGB}{241,243,245}
\newenvironment{Shaded}{\begin{snugshade}}{\end{snugshade}}
\newcommand{\AlertTok}[1]{\textcolor[rgb]{0.68,0.00,0.00}{#1}}
\newcommand{\AnnotationTok}[1]{\textcolor[rgb]{0.37,0.37,0.37}{#1}}
\newcommand{\AttributeTok}[1]{\textcolor[rgb]{0.40,0.45,0.13}{#1}}
\newcommand{\BaseNTok}[1]{\textcolor[rgb]{0.68,0.00,0.00}{#1}}
\newcommand{\BuiltInTok}[1]{\textcolor[rgb]{0.00,0.23,0.31}{#1}}
\newcommand{\CharTok}[1]{\textcolor[rgb]{0.13,0.47,0.30}{#1}}
\newcommand{\CommentTok}[1]{\textcolor[rgb]{0.37,0.37,0.37}{#1}}
\newcommand{\CommentVarTok}[1]{\textcolor[rgb]{0.37,0.37,0.37}{\textit{#1}}}
\newcommand{\ConstantTok}[1]{\textcolor[rgb]{0.56,0.35,0.01}{#1}}
\newcommand{\ControlFlowTok}[1]{\textcolor[rgb]{0.00,0.23,0.31}{#1}}
\newcommand{\DataTypeTok}[1]{\textcolor[rgb]{0.68,0.00,0.00}{#1}}
\newcommand{\DecValTok}[1]{\textcolor[rgb]{0.68,0.00,0.00}{#1}}
\newcommand{\DocumentationTok}[1]{\textcolor[rgb]{0.37,0.37,0.37}{\textit{#1}}}
\newcommand{\ErrorTok}[1]{\textcolor[rgb]{0.68,0.00,0.00}{#1}}
\newcommand{\ExtensionTok}[1]{\textcolor[rgb]{0.00,0.23,0.31}{#1}}
\newcommand{\FloatTok}[1]{\textcolor[rgb]{0.68,0.00,0.00}{#1}}
\newcommand{\FunctionTok}[1]{\textcolor[rgb]{0.28,0.35,0.67}{#1}}
\newcommand{\ImportTok}[1]{\textcolor[rgb]{0.00,0.46,0.62}{#1}}
\newcommand{\InformationTok}[1]{\textcolor[rgb]{0.37,0.37,0.37}{#1}}
\newcommand{\KeywordTok}[1]{\textcolor[rgb]{0.00,0.23,0.31}{#1}}
\newcommand{\NormalTok}[1]{\textcolor[rgb]{0.00,0.23,0.31}{#1}}
\newcommand{\OperatorTok}[1]{\textcolor[rgb]{0.37,0.37,0.37}{#1}}
\newcommand{\OtherTok}[1]{\textcolor[rgb]{0.00,0.23,0.31}{#1}}
\newcommand{\PreprocessorTok}[1]{\textcolor[rgb]{0.68,0.00,0.00}{#1}}
\newcommand{\RegionMarkerTok}[1]{\textcolor[rgb]{0.00,0.23,0.31}{#1}}
\newcommand{\SpecialCharTok}[1]{\textcolor[rgb]{0.37,0.37,0.37}{#1}}
\newcommand{\SpecialStringTok}[1]{\textcolor[rgb]{0.13,0.47,0.30}{#1}}
\newcommand{\StringTok}[1]{\textcolor[rgb]{0.13,0.47,0.30}{#1}}
\newcommand{\VariableTok}[1]{\textcolor[rgb]{0.07,0.07,0.07}{#1}}
\newcommand{\VerbatimStringTok}[1]{\textcolor[rgb]{0.13,0.47,0.30}{#1}}
\newcommand{\WarningTok}[1]{\textcolor[rgb]{0.37,0.37,0.37}{\textit{#1}}}

\providecommand{\tightlist}{%
  \setlength{\itemsep}{0pt}\setlength{\parskip}{0pt}}\usepackage{longtable,booktabs,array}
\usepackage{calc} % for calculating minipage widths
% Correct order of tables after \paragraph or \subparagraph
\usepackage{etoolbox}
\makeatletter
\patchcmd\longtable{\par}{\if@noskipsec\mbox{}\fi\par}{}{}
\makeatother
% Allow footnotes in longtable head/foot
\IfFileExists{footnotehyper.sty}{\usepackage{footnotehyper}}{\usepackage{footnote}}
\makesavenoteenv{longtable}
\usepackage{graphicx}
\makeatletter
\def\maxwidth{\ifdim\Gin@nat@width>\linewidth\linewidth\else\Gin@nat@width\fi}
\def\maxheight{\ifdim\Gin@nat@height>\textheight\textheight\else\Gin@nat@height\fi}
\makeatother
% Scale images if necessary, so that they will not overflow the page
% margins by default, and it is still possible to overwrite the defaults
% using explicit options in \includegraphics[width, height, ...]{}
\setkeys{Gin}{width=\maxwidth,height=\maxheight,keepaspectratio}
% Set default figure placement to htbp
\makeatletter
\def\fps@figure{htbp}
\makeatother

\usepackage{booktabs}
\usepackage{longtable}
\usepackage{array}
\usepackage{multirow}
\usepackage{wrapfig}
\usepackage{float}
\usepackage{colortbl}
\usepackage{pdflscape}
\usepackage{tabu}
\usepackage{threeparttable}
\usepackage{threeparttablex}
\usepackage[normalem]{ulem}
\usepackage{makecell}
\usepackage{xcolor}
\KOMAoption{captions}{tableheading}
\makeatletter
\makeatother
\makeatletter
\makeatother
\makeatletter
\@ifpackageloaded{caption}{}{\usepackage{caption}}
\AtBeginDocument{%
\ifdefined\contentsname
  \renewcommand*\contentsname{Table of contents}
\else
  \newcommand\contentsname{Table of contents}
\fi
\ifdefined\listfigurename
  \renewcommand*\listfigurename{List of Figures}
\else
  \newcommand\listfigurename{List of Figures}
\fi
\ifdefined\listtablename
  \renewcommand*\listtablename{List of Tables}
\else
  \newcommand\listtablename{List of Tables}
\fi
\ifdefined\figurename
  \renewcommand*\figurename{Figure}
\else
  \newcommand\figurename{Figure}
\fi
\ifdefined\tablename
  \renewcommand*\tablename{Table}
\else
  \newcommand\tablename{Table}
\fi
}
\@ifpackageloaded{float}{}{\usepackage{float}}
\floatstyle{ruled}
\@ifundefined{c@chapter}{\newfloat{codelisting}{h}{lop}}{\newfloat{codelisting}{h}{lop}[chapter]}
\floatname{codelisting}{Listing}
\newcommand*\listoflistings{\listof{codelisting}{List of Listings}}
\makeatother
\makeatletter
\@ifpackageloaded{caption}{}{\usepackage{caption}}
\@ifpackageloaded{subcaption}{}{\usepackage{subcaption}}
\makeatother
\makeatletter
\@ifpackageloaded{tcolorbox}{}{\usepackage[many]{tcolorbox}}
\makeatother
\makeatletter
\@ifundefined{shadecolor}{\definecolor{shadecolor}{rgb}{.97, .97, .97}}
\makeatother
\makeatletter
\makeatother
\ifLuaTeX
  \usepackage{selnolig}  % disable illegal ligatures
\fi
\IfFileExists{bookmark.sty}{\usepackage{bookmark}}{\usepackage{hyperref}}
\IfFileExists{xurl.sty}{\usepackage{xurl}}{} % add URL line breaks if available
\urlstyle{same} % disable monospaced font for URLs
\hypersetup{
  pdftitle={Data cleaning and preprocessing},
  colorlinks=true,
  linkcolor={blue},
  filecolor={Maroon},
  citecolor={Blue},
  urlcolor={Blue},
  pdfcreator={LaTeX via pandoc}}

\title{Data cleaning and preprocessing}
\author{}
\date{}

\begin{document}
\maketitle
\ifdefined\Shaded\renewenvironment{Shaded}{\begin{tcolorbox}[interior hidden, enhanced, borderline west={3pt}{0pt}{shadecolor}, sharp corners, breakable, frame hidden, boxrule=0pt]}{\end{tcolorbox}}\fi

\hypertarget{read-data-and-load-packages}{%
\section{Read Data and Load
Packages}\label{read-data-and-load-packages}}

\begin{Shaded}
\begin{Highlighting}[]
\CommentTok{\# load important packages}
\NormalTok{pacman}\SpecialCharTok{::}\FunctionTok{p\_load}\NormalTok{(tidyverse, naniar, dlookr, DataExplorer, janitor, tidymodels)}


\CommentTok{\#read in the csv file and clean the column names}
\NormalTok{vgsales\_df }\OtherTok{\textless{}{-}} \FunctionTok{read\_csv}\NormalTok{(}\StringTok{"datasets/vgsales{-}12{-}4{-}2019.csv"}\NormalTok{, }\AttributeTok{show\_col\_types =}\NormalTok{ F) }\SpecialCharTok{\%\textgreater{}\%}   
  \CommentTok{\# start with cleaning column names into a consistent convention}
  \FunctionTok{clean\_names}\NormalTok{()}
\end{Highlighting}
\end{Shaded}

\hypertarget{explore-missingness}{%
\section{Explore missingness}\label{explore-missingness}}

\begin{Shaded}
\begin{Highlighting}[]
\CommentTok{\#this code is used to show the percentage of missing values in each variable}
\FunctionTok{gg\_miss\_var}\NormalTok{(vgsales\_df,}\AttributeTok{show\_pct =}\NormalTok{ T)}\SpecialCharTok{+}
  \FunctionTok{labs}\NormalTok{(}\AttributeTok{title =} \StringTok{"\% missingness by variable"}\NormalTok{)}
\end{Highlighting}
\end{Shaded}

\begin{figure}[H]

{\centering \includegraphics{data-cleainig_files/figure-pdf/unnamed-chunk-2-1.pdf}

}

\end{figure}

\hypertarget{create-an-interactive-report-for-data-exploration}{%
\section{create an interactive report for data
exploration}\label{create-an-interactive-report-for-data-exploration}}

\begin{Shaded}
\begin{Highlighting}[]
\CommentTok{\# create a report using the data in vgsales\_df, output the report in html format, name the report vgsales\_eda.html, and save the report in the current working directory}
\FunctionTok{create\_report}\NormalTok{(}
  \AttributeTok{data =}\NormalTok{ vgsales\_df, }
  \AttributeTok{output\_format =} \StringTok{"html\_document"}\NormalTok{, }
  \AttributeTok{output\_file =} \StringTok{"vgsales\_eda.html"}\NormalTok{, }
  \AttributeTok{output\_dir =} \StringTok{"./"}
\NormalTok{)}
\end{Highlighting}
\end{Shaded}

\hypertarget{some-eda}{%
\section{Some EDA}\label{some-eda}}

\begin{Shaded}
\begin{Highlighting}[]
\CommentTok{\# find out the count of genre by year(which year which genre sold more)}
\NormalTok{vgsales\_df }\SpecialCharTok{\%\textgreater{}\%} 
  \FunctionTok{count}\NormalTok{(year, genre, }\AttributeTok{sort =}\NormalTok{ T) }\SpecialCharTok{\%\textgreater{}\%} 
  \CommentTok{\# filter first 10 results}
  \FunctionTok{slice\_head}\NormalTok{(}\AttributeTok{n =} \DecValTok{10}\NormalTok{)}
\end{Highlighting}
\end{Shaded}

\begin{verbatim}
# A tibble: 10 x 3
    year genre         n
   <dbl> <chr>     <int>
 1  2014 Misc       1273
 2  2013 Misc        936
 3  2009 Misc        844
 4  2012 Misc        721
 5  2010 Misc        673
 6  2011 Misc        671
 7  2009 Action      572
 8  2008 Misc        496
 9  2009 Adventure   481
10  2011 Action      480
\end{verbatim}

\hypertarget{data-cleaning-pipeline}{%
\section{Data Cleaning pipeline}\label{data-cleaning-pipeline}}

\begin{Shaded}
\begin{Highlighting}[]
\CommentTok{\# prepare data for modeling with sales as the target variable}


\NormalTok{vgsales\_df\_clean }\OtherTok{\textless{}{-}}\NormalTok{ vgsales\_df }\SpecialCharTok{\%\textgreater{}\%}
  \CommentTok{\# remove columns that are not needed for analysis }
  \CommentTok{\# please refer to the above eda report on the missing variable part}
  \FunctionTok{select}\NormalTok{(}
    \SpecialCharTok{{-}}\FunctionTok{c}\NormalTok{(basename, url, img\_url, vg\_chartz\_score, vgchartzscore, }
\NormalTok{       user\_score, total\_shipped, critic\_score, last\_update)}
\NormalTok{    ) }\SpecialCharTok{\%\textgreater{}\%} 
   \CommentTok{\# change character columns to factors}
  \FunctionTok{mutate\_if}\NormalTok{(is.character, as.factor) }\SpecialCharTok{\%\textgreater{}\%} 
  \CommentTok{\# reduce factor levels of genre, esrb\_rating, platform, developer}
  \CommentTok{\# also refer to the charts in the eda report to see the most common levels in each factor variable}
  \FunctionTok{mutate}\NormalTok{(}
    \AttributeTok{genre =} \FunctionTok{fct\_lump\_n}\NormalTok{(genre, }\AttributeTok{n =} \DecValTok{10}\NormalTok{, }\AttributeTok{other\_level =} \StringTok{"Other"}\NormalTok{),}
    \AttributeTok{esrb\_rating =} \FunctionTok{fct\_lump\_n}\NormalTok{(esrb\_rating, }\AttributeTok{n =} \DecValTok{4}\NormalTok{, }\AttributeTok{other\_level =} \StringTok{"Other"}\NormalTok{),}
    \AttributeTok{platform =} \FunctionTok{fct\_lump\_n}\NormalTok{(platform, }\AttributeTok{n =} \DecValTok{10}\NormalTok{, }\AttributeTok{other\_level =} \StringTok{"Other"}\NormalTok{),}
    \AttributeTok{developer =} \FunctionTok{fct\_lump\_n}\NormalTok{(developer, }\AttributeTok{n =} \DecValTok{10}\NormalTok{, }\AttributeTok{other\_level =} \StringTok{"Other"}\NormalTok{),}
    \AttributeTok{publisher =} \FunctionTok{fct\_lump\_n}\NormalTok{(publisher, }\AttributeTok{n =} \DecValTok{10}\NormalTok{, }\AttributeTok{other\_level =} \StringTok{"Other"}\NormalTok{)}
\NormalTok{    ) }\SpecialCharTok{\%\textgreater{}\%} 
  \CommentTok{\# I want to create a variable (sales) that combines all sales features with sales aspect}
  \CommentTok{\# before that we should replace NA\textquotesingle{}s with zero to for mutate to work}
  \FunctionTok{replace\_na}\NormalTok{(}\FunctionTok{list}\NormalTok{(}\AttributeTok{global\_sales =} \DecValTok{0}\NormalTok{,}
                  \AttributeTok{na\_sales =} \DecValTok{0}\NormalTok{,}
                  \AttributeTok{pal\_sales =} \DecValTok{0}\NormalTok{,}
                  \AttributeTok{jp\_sales =} \DecValTok{0}\NormalTok{, }
                  \AttributeTok{other\_sales =} \DecValTok{0}\NormalTok{)) }\SpecialCharTok{\%\textgreater{}\%} 
  \CommentTok{\# create a variable sales}
  \FunctionTok{mutate}\NormalTok{(}\AttributeTok{sales =}\NormalTok{ global\_sales }\SpecialCharTok{+}\NormalTok{ pal\_sales }\SpecialCharTok{+}\NormalTok{ jp\_sales }\SpecialCharTok{+}\NormalTok{ other\_sales) }\SpecialCharTok{\%\textgreater{}\%} 
  \CommentTok{\# deselect the variables that made sales, }
  \CommentTok{\# also deselect status because it contains only one value input(1) confirm with \textasciigrave{}count(vgsales\_df, status)\textasciigrave{}}
  \CommentTok{\# name and rank also dont seem to add value our model}
  \FunctionTok{select}\NormalTok{(}\SpecialCharTok{{-}}\FunctionTok{c}\NormalTok{(other\_sales, jp\_sales, pal\_sales, na\_sales, global\_sales, status, rank, name)) }\SpecialCharTok{\%\textgreater{}\%} 
  \CommentTok{\# remove rows with missing values }
  \CommentTok{\# for simplicity, am choosing removing the rows with NA\textquotesingle{}s}
  \CommentTok{\# we could the knn algorithm to impute NA\textquotesingle{}s in the feature engineering recipe}
  \FunctionTok{drop\_na}\NormalTok{() }\SpecialCharTok{\%\textgreater{}\%} 
   \CommentTok{\# remove duplicate rows}
  \FunctionTok{distinct}\NormalTok{()}
\end{Highlighting}
\end{Shaded}

The above cleaning pipeline can be changed to a function to clean future
sorces of data

\hypertarget{deep-eda-of-cleaned-data}{%
\section{Deep EDA of cleaned data}\label{deep-eda-of-cleaned-data}}

\begin{Shaded}
\begin{Highlighting}[]
\CommentTok{\# create a deep eda web report of the cleaned data with this code}
\CommentTok{\# use the arrows in each analysis part to see the graphs and statistics}
\FunctionTok{eda\_web\_report}\NormalTok{(}
  \AttributeTok{.data =}\NormalTok{ vgsales\_df\_clean, }
  \AttributeTok{target =} \StringTok{"sales"}\NormalTok{, }
  \AttributeTok{output\_file =} \StringTok{"vgsales\_deep\_dive.html"}\NormalTok{, }
  \AttributeTok{output\_dir =} \StringTok{"./"}
\NormalTok{)}
\end{Highlighting}
\end{Shaded}

\hypertarget{data-splicing}{%
\section{Data Splicing}\label{data-splicing}}

\begin{Shaded}
\begin{Highlighting}[]
\CommentTok{\# split data into training and testing sets, allocating 70\% to the training set}
\NormalTok{sales\_split }\OtherTok{\textless{}{-}} \FunctionTok{initial\_split}\NormalTok{(vgsales\_df\_clean, }\AttributeTok{prop =}\NormalTok{ .}\DecValTok{70}\NormalTok{)}

\CommentTok{\# print sales\_split}
\NormalTok{sales\_split}
\end{Highlighting}
\end{Shaded}

\begin{verbatim}
<Training/Testing/Total>
<11992/5140/17132>
\end{verbatim}

\hypertarget{feature-engineering-pipeline}{%
\section{Feature Engineering
Pipeline}\label{feature-engineering-pipeline}}

\begin{Shaded}
\begin{Highlighting}[]
\CommentTok{\# prepare a feature engineering pipeline}
\NormalTok{my\_recipe }\OtherTok{\textless{}{-}} \FunctionTok{recipe}\NormalTok{(sales }\SpecialCharTok{\textasciitilde{}}\NormalTok{ ., }\AttributeTok{data =} \FunctionTok{training}\NormalTok{(sales\_split)) }\SpecialCharTok{\%\textgreater{}\%} 
  \CommentTok{\# center and scale the year variable}
  \FunctionTok{step\_normalize}\NormalTok{(year) }\SpecialCharTok{\%\textgreater{}\%} 
  \CommentTok{\# create dummy variables for all the nominal variables with one{-}hot encoding}
  \FunctionTok{step\_dummy}\NormalTok{(}\FunctionTok{all\_nominal}\NormalTok{(), }\AttributeTok{one\_hot =} \ConstantTok{TRUE}\NormalTok{)}

\CommentTok{\# print recipe}
\NormalTok{my\_recipe}
\end{Highlighting}
\end{Shaded}

\begin{verbatim}
Recipe

Inputs:

      role #variables
   outcome          1
 predictor          6

Operations:

Centering and scaling for year
Dummy variables from all_nominal()
\end{verbatim}

\begin{Shaded}
\begin{Highlighting}[]
\CommentTok{\# Preprocess training data}
\NormalTok{training\_data\_processed }\OtherTok{\textless{}{-}}\NormalTok{ my\_recipe }\SpecialCharTok{\%\textgreater{}\%} 
  \CommentTok{\# train recipe with training data}
  \FunctionTok{prep}\NormalTok{(}\FunctionTok{training}\NormalTok{(sales\_split)) }\SpecialCharTok{\%\textgreater{}\%} 
  \CommentTok{\# setting new data to NULL returns the preprocessed training data}
  \FunctionTok{bake}\NormalTok{(}\AttributeTok{new\_data =} \ConstantTok{NULL}\NormalTok{)}
  

\CommentTok{\# preprocess testing data}
\NormalTok{testing\_data\_processed }\OtherTok{\textless{}{-}}\NormalTok{ my\_recipe }\SpecialCharTok{\%\textgreater{}\%} 
  \CommentTok{\# train recipe with training data}
  \FunctionTok{prep}\NormalTok{(}\FunctionTok{training}\NormalTok{(sales\_split)) }\SpecialCharTok{\%\textgreater{}\%} 
  \FunctionTok{bake}\NormalTok{(}\AttributeTok{new\_data =} \FunctionTok{testing}\NormalTok{(sales\_split))}
  
\CommentTok{\# we can now proceed to specify a model to fit the data with the parsnip package}
\end{Highlighting}
\end{Shaded}




\end{document}
